% %% %%%%%%%%%%%%%%%%%%%%%%%%%%%%%%%%%%%%%%%%%%%%%%%%%%%%%%%%%%
% step-2.tex
%
% Author:  Mauricio Matamoros
% License: MIT
%
% %% %%%%%%%%%%%%%%%%%%%%%%%%%%%%%%%%%%%%%%%%%%%%%%%%%%%%%%%%%%
%!TEX root = ../practica.tex
%!TEX root = ../references.bib

% CHKTEX-FILE 1
% CHKTEX-FILE 13
% CHKTEX-FILE 46
\subsection{Paso 2: Configuración de comunicaciones \IIC}%
\label{sec:step3}
Primero ha de configurarse la Raspberry Pi para funcionar como dispositivo maestro o \emph{master} en el bus \IIC.
Para esto, inicie la utilidad de configuración de la Raspberry Pi con el comando

\begin{Verbatim}
# raspi-config
\end{Verbatim}

\noindent y seleccione la opción 5: Opciones de Interfaz (\emph{Interfacing Options}) y active la opción \texttt{P5} para habilitar el \IIC.

A continuación, verifique que el puerto \IIC no se encuentre en la lista negra.
Edite el archivo \\\texttt{/etc/modprobe.d/raspi-blacklist.conf} y revise que la línea \texttt{blacklist spi-bcm2708} esté comentada con \#.

% $ cat /etc/modprobe.d/raspi-blacklist.conf
\begin{lstlisting}[
	language=conf,
	caption={/etc/modprobe.d/raspi-blacklist.conf},
	label={lst:raspi-blacklist.conf},
	numbers=none
]
# blacklist spi and i2c by default (many users don't need them)
# blacklist i2c-bcm2708
\end{lstlisting}

Como paso siguiente, se habilita la carga del driver \IIC.
Esto se logra agregando la línea \texttt{i2c-dev} al final del archivo \texttt{/etc/modules} si esta no se encuentra ya allí.

Por último, se instalan los paquetes que permiten la comunicación mediante el bus \IIC y se habilita al usuario predeterminado \emph{pi} (o cualquier otro que se esté usando) para acceder al recurso.

\begin{Verbatim}
# apt-get install i2c-tools python-smbus
# adduser pi i2c
\end{Verbatim}

Reinicie la Raspberry Pi y pruebe la configuración ejecutando \texttt{i2cdetect -y 1} para buscar dispositivos conectados al bus \IIC.
Debería ver una salida como la siguiente:

\begin{Verbatim}
\$ i2cdetect -y 1
     0  1  2  3  4  5  6  7  8  9  a  b  c  d  e  f
00:          -- -- -- -- -- -- -- -- -- -- -- --
10: -- -- -- -- -- -- -- -- -- -- -- -- -- -- --
20: -- -- -- -- -- -- -- -- -- -- -- -- -- -- --
30: -- -- -- -- -- -- -- -- -- -- -- -- -- -- --
40: -- -- -- -- -- -- -- -- -- -- -- -- -- -- --
50: -- -- -- -- -- -- -- -- -- -- -- -- -- -- --
60: -- -- -- -- -- -- -- -- -- -- -- -- -- -- --
70: -- -- -- -- -- -- -- --
\end{Verbatim}
