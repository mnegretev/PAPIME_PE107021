% %% %%%%%%%%%%%%%%%%%%%%%%%%%%%%%%%%%%%%%%%%%%%%%%%%%%%%%%%%%%
% intro-control-pd.tex
%
% Author:  Mauricio Matamoros
% License: MIT
%
% %% %%%%%%%%%%%%%%%%%%%%%%%%%%%%%%%%%%%%%%%%%%%%%%%%%%%%%%%%%%
%!TEX root = ../practica.tex
%!TEX root = ../references.bib

% CHKTEX-FILE 1
% CHKTEX-FILE 13
% CHKTEX-FILE 46

\subsubsection{Control derivativo}%
\label{sec:control-d}
Un control es derivativo cuando la salida $v(t)$ del controlador es proporcional a la derivada del error $e(t)$:

\begin{equation*}
v(t) = K_{D}\;\frac{d\;e(t)}{dt}
\end{equation*}

\noindent o en el dominio de la frecuencia:

\begin{equation*}
V(s) = K_{D}\;s\;E(s)
\end{equation*}

\noindent por lo tanto

\begin{equation}
G_{c}(s) = \frac{V(s)}{E(s)} = K_{D}s
\label{eqn:ctrl-d}
\end{equation}

Debido a que la derivada del error respecto al tiempo es la velocidad del error, un control derivativo se anticipará a los errores antes de que estos se produzcan, amortiguando oscilaciones.
Sin embargo, un controlador D es insensible a errores de estado estable, por lo que suele acompañarse de otros elementos que sí corrijan el error de estado estable.

Cuando se combina con un controlador tipo proporcional para formar un control PD \Citep{Hernandez2010}:
\begin{itemize}[noitemsep]
	\item El amortiguamiento se incrementa.
	\item El máximo pico de sobreimpulso se reduce.
	\item El tiempo de elevación experimenta pequeños cambios.
	\item Se mejoran el margen de ganancia y el margen de fase.
	\item El error de estado estable presenta pequeños cambios.
	\item El tipo de sistema permanece igual.
\end{itemize}

