% %% %%%%%%%%%%%%%%%%%%%%%%%%%%%%%%%%%%%%%%%%%%%%%%%%%%%%%%%%%%
% intro-control-p.tex
%
% Author:  Mauricio Matamoros
% License: MIT
%
% %% %%%%%%%%%%%%%%%%%%%%%%%%%%%%%%%%%%%%%%%%%%%%%%%%%%%%%%%%%%
%!TEX root = ../practica.tex
%!TEX root = ../references.bib

% CHKTEX-FILE 1
% CHKTEX-FILE 13
% CHKTEX-FILE 46

\subsubsection{Control proporcional}%
\label{sec:control-p}
Un control es proporcional cuando la salida $v(t)$ del controlador es proporcional al error $e(t)$:
\begin{equation*}
v(t ) = K_{P} e(t)
\end{equation*}

\noindent o en el dominio de la frecuencia:

\begin{equation*}
V(s) = K_{P} E(s)
\end{equation*}

\noindent por lo tanto

\begin{equation}
G_{c}(s) = \frac{V(s)}{E(s)} = K_{P}
\label{eqn:ctrl-p}
\end{equation}

Al otorgar una señal de entrada proporcional al error, éste tipo de control opera como un amplificador de error que, al corregirse, afecta negativamente la respuesta transitoria del sistema.
Es por esto que, a pesar de que un controlador P es fácil de implementar y ajustar, suele venir acompañado de otros elementos que compensen la amplificación del error y las variaciones transitorias.

Cuando se utiliza un control proporcional~\Citep{Hernandez2010}:
\begin{itemize}[noitemsep]
	\item El tiempo de elevación experimenta una pequeña reducción.
	\item El máximo pico de sobreimpulso se incrementa.
	\item El amortiguamiento se reduce.
	\item El tiempo de asentamiento cambia en pequeña proporción.
	\item El error de estado estable disminuye con incrementos de ganancia.
	\item El tipo de sistema permanece igual.
\end{itemize}

