% %% %%%%%%%%%%%%%%%%%%%%%%%%%%%%%%%%%%%%%%%%%%%%%%%%%%%%%%%%%%
% appspecs.tex
%
% Author:  Mauricio Matamoros
% License: MIT
%
% %% %%%%%%%%%%%%%%%%%%%%%%%%%%%%%%%%%%%%%%%%%%%%%%%%%%%%%%%%%%

%!TEX root = ../main.tex
%!TEX root = ../references.bib

\section{Programas}%
\label{sec:programs}

Integre el código del Programa 1 en un archivo python llamado \texttt{led\_manager.py} y que ofrezca las siguientes funciones:
\begin{enumerate}
	\item{} [2 pts] Encendido del del 1--7 al presionar el boton adecuado.

	\item{} [2 pts] Desplegado de la marquesina izquierda al presionar el boton adecuado.\footnote{Una marquesina izquiera muestra el corrimiento circular hacia la izquierda del led menos significativo (i.e.~de derecha o menos significativo a izquierda o más significativo), manteniendo únicamente un led encendido en todo momento.}

	\item{} [2 pts] Desplegado de la marquesina derecha al presionar el boton adecuado.\footnote{Una marquesina derecha muestra el corrimiento circular hacia la derecha del led más significativo (i.e.~izquierda o más significativo a de derecha o menos significativo), manteniendo únicamente un led encendido en todo momento.}

	\item{} [2 pts] Desplegado de la marquesina tipo ping-pong al presionar el boton adecuado.\footnote{Una marquesina tipo ping-pong o de rebote muestra el corrimiento hacia la derecha del led más significativo hasta alcanzar el led menos significativo, momento en el que se invierte la dirección del corrimiento y así sucesivamente, dando la impresión de que el led \enquote{rebota} en los extremos.}

	\item{} [2 pts] Desplegado del dígito correcto en el display de 7 segmentos al presionar el boton correspondiente
\end{enumerate}
\section{Especificaciones técnicas de los programas}%
\label{sec:programs-specs}
\begin{itemize}[noitemsep]
	\item No utilice paquetes adicionales.
	\item El código deberá ser ejecutable con Python versión 3.5 o posterior.
	\item Todos los programas deberán comenzar con la línea de intérprete o \emph{she-bang} correspondiente
	\item Todos los programas deberán tener el nombre del autor de la forma:

\begin{lstlisting}[language=python]
# Author: Nombre del Alumno
\end{lstlisting}
	\item En los videos-evidencia deberá observarse claramente cómo el alumno controla remotamente desde la interfaz de usuario al simulador de la RaspBerry Pi (ej.~desde su celular o desde otra máquina virtual).

	\item Incluya sólo los videos y el código fuente de los programas \textbi{sin librerías ni paquetes}.
	\item Los archivos de código python deberán estar en raíz \texttt{./}.
	\item Los videos-evidencia deberán estar en el subdirectorio \texttt{./vid/}.
	\item Los videos-evidencia deberán durar no más de 60 segundos, incluir sólo la ventana del simulador y contar únicamente con \emph{stream} de video comprimido con \emph{codec} h.264 a \(15fps\) con una resolución máxima de \(1280 \times 720\) y con un tamaño máximo de 3MB por archivo (velocidad de datos aproximada de \(1500kbps\))\footnote{\texttt{ffmpeg -i input -an -vf scale=-1:720 -c:v libx264 -crf 28 -r 15 -preset veryslow hicm\_osblink8.mp4}}.
	\item Los entregables deberán estar empaquetados en un archivo comprimido de nombre \texttt{[prefijo]\_p02} donde \texttt{[prefijo]\_p02} corresponde a los primeros 4 caracteres de la CURP del alumno, por ejemplo \texttt{hicm\_p01.zip}.
	Los formatos aceptables son \emph{7z}, \emph{rar}, \emph{tar.bz2}, \emph{tar.gz} y \emph{zip}.
\end{itemize}
