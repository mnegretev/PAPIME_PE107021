% %% %%%%%%%%%%%%%%%%%%%%%%%%%%%%%%%%%%%%%%%%%%%%%%%%%%%%%%%%%%
% step-3.tex
%
% Author:  Mauricio Matamoros
% License: MIT
%
% %% %%%%%%%%%%%%%%%%%%%%%%%%%%%%%%%%%%%%%%%%%%%%%%%%%%%%%%%%%%

%!TEX root = ../main.tex
%!TEX root = ../references.bib

\subsection{Paso 3: Display de siete segmentos}%
\label{sec:step4}
El código mostrado en \Cref{src:bcd} muestra cómo se operaría un display de siete segmentos mediante una controladora TTL 74LS47 utilizando la Raspberry Pi.

\smallskip
\lstinputlisting[%
	language=Python,
	linerange={19-51}, % chktex 8
	caption={\texttt{bcd.py}},
	label={src:bcd}
]{src/bcd.py}
\smallskip

Estudie el código y véalo en funcionamiento, ejecutándolo de la siguiente manera:
\begin{Verbatim}[fontsize=\footnotesize]
./bcd.py
\end{Verbatim}

