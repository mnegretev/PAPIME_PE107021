% %% %%%%%%%%%%%%%%%%%%%%%%%%%%%%%%%%%%%%%%%%%%%%%%%%%%%%%%%%%%
% materials.tex
%
% Author:  Mauricio Matamoros
% License: MIT
%
% %% %%%%%%%%%%%%%%%%%%%%%%%%%%%%%%%%%%%%%%%%%%%%%%%%%%%%%%%%%%

%!TEX root = ../main.tex
%!TEX root = ../references.bib

\section{Material}%
\label{sec:material}
\begin{enumerate}[noitemsep]
	\item Plataforma:
	\begin{itemize}[nosep]
		\item Computadora con sistema operativo Linux, o
		\item Máquina virtual con sistema operativo Raspbian, o
		\item Raspberry Pi con sistema operativo Raspbian
	\end{itemize}
	\item Interprete de Python 3.5 instalado.
	\item Código del programa anterior.
\end{enumerate}

\begin{importantbox}{\bfseries ADVERTENCIA}
Cuando se configura un sistema Linux como punto de acceso inalámbrico con DHCP, éste pierde la capacidad de utilizar la tarjeta inalámbrica para conectarse a internet via inalámbrica.

Por este motivo se recomienda que los programas se prueben en una máquina virtual o en una Raspberry Pi si no se sabe cómo revertir este proceso.
\end{importantbox}

\textbf{Nota:} Si se cuenta con una Raspberry Pi sin WiFi integrado (e.j~Raspberry Pi2), se precisará de un adaptador WiFi USB compatible para la misma.
