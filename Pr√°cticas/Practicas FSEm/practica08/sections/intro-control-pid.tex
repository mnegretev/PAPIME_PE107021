% %% %%%%%%%%%%%%%%%%%%%%%%%%%%%%%%%%%%%%%%%%%%%%%%%%%%%%%%%%%%
% intro-control-pid.tex
%
% Author:  Mauricio Matamoros
% License: MIT
%
% %% %%%%%%%%%%%%%%%%%%%%%%%%%%%%%%%%%%%%%%%%%%%%%%%%%%%%%%%%%%
%!TEX root = ../practica.tex
%!TEX root = ../references.bib

% CHKTEX-FILE 1
% CHKTEX-FILE 13
% CHKTEX-FILE 46

\subsubsection{Control PID digital}%
\label{sec:control-pid}
Un control \emph{proporcional-integral-derivativo} o PID incorpora las mejores características de los controles PI y PD, motivo por el cual ha sido uno de las técnicas de control continuo más utilizados en la industria por más de siete décadas \citep{Hernandez2010,Odwyer2009,Ogata2003}.

Con base en las ecuaciones \Cref{eqn:ctrl-p,sec:control-i,eqn:ctrl-d} el controlador sería de la forma:

\begin{equation}
G_{c}(s) = \frac{V(s)}{E(s)} =
	K_{P} + \frac{K_I}{s} + K_{D}s
\end{equation}

Sin embargo, como se mencionó anteriormente en la \Cref{intro-control}, integrar señales en tiempo contínuo es computacionalmente muy costoso.
Además, la señal del sensor pasará por un convertidor analógico-digital que discretizará la señal, por lo que tiene más sentido implementar un controlador discreto.
Luego entonces, interesa conocer el valor (también discreto) que se proporcionará a la planta para obtener la señal deseada en el sensor.
En el contínuo esta sería:

\begin{equation*}
v(t) = K_{P}\,e(t) + {K_I}\int e(t)\,dt+ K_{D}\frac{d\,e(t)}{dt}
\label{eqn:v-pid}
\end{equation*}

Si se muestrea el error a una intervalos de tiempo constante con frecunecia lo suficientemente alta, es posible convertir la \cref{eqn:v-pid} en una ecuación en diferencias sobre la $k$-ésima muestra tomada:

\begin{equation*}
v[k] =
		K_{P}\,e[k] +
		K_{I}\sum^{k}_{j=0} e[j]+
		K_{D}\big(e[k] - e[k-1]\big)
\end{equation*}

\noindent donde

\begin{align*}
	&e[k]                     && \text{el error actual}\\
	&e[k-1]                   && \text{el error anterior}
\end{align*}

Además, si se considera $P_{I}[k] = K_{I}\sum^{k}_{j=0} e[j]$ entonces:

\begin{align*}
P_{I}[k]
	&= K_{I}\sum^{k}_{j=0} e[j]\\
	&= K_{I}\,e[k] + P_{i}[k-1]
\end{align*}

Así, aproximar un control digital para un sistema simple puede ser tan sencillo como almacenar en memoria los últimos valores sensados del error y el acumulado de la componente integral del PID, lo que reduce el cómputo a sólo unas cuantas operaciones aritméticas simples: sumas, restas y productos.
No obstante, es necesario recordar que ésto no funcionará con sistemas que tengan una dinámica compleja, particularmente los no-lineales.

