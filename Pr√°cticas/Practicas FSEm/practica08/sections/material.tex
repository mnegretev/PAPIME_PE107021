% CHKTEX-FILE 1
% CHKTEX-FILE 13
% CHKTEX-FILE 46
%!TEX root = ../practica.tex
%!TEX root = ../references.bib

%% %%%%%%%%%%%%%%%%%%%%%%%%%%%%%%%%%%%%%%%%%%%%%%%%%%%%%%%%%%%%%%%%%%
%
% Material
%
%% %%%%%%%%%%%%%%%%%%%%%%%%%%%%%%%%%%%%%%%%%%%%%%%%%%%%%%%%%%%%%%%%%%
\section{Material}%
\label{sec:material}
Se asume que el alumno cuenta con un una Raspberry Pi con sistema operativo Raspbian e interprete de Python instalado. Se aconseja encarecidamente el uso de \textit{git} como programa de control de versiones.

Además, el alumno necesitará los alambrados de la
\href{https://github.com/kyordhel/FSEm/tree/master/practica06}{Práctica 6}
y la
\href{https://github.com/kyordhel/FSEm/tree/master/practica07}{Práctica 7}.

\subsection{Lista de componentes}%
\label{sec:material-component-list}

\begin{itemize}[noitemsep]
	\item 1 Arduino UNO o Arduino Mega
	\item 1 sensor de temperatura LM35 en encapsulado TO-220 o TO-92
	\item 1 TRIAC BT138 o BT139
	\item 4 diodos 1N4007 o puente rectificador equivalente
	\item 2 diodos 1N914
	\item 1 optoacoplador MOC 3021
	\item 1 optoacoplador 4N25
	\item 1 foco incandescente (NO AHORRADOR NI LED)
	\item 1 resistencia de 15k$\Omega$
	\item 1 resistencia de 18k$\Omega$
	\item 1 resistencia de 12k$\Omega$\footnotemark
	\item 3 resistencia de 10k$\Omega$
	\item 2 resistencia de 4k7$\Omega$
	\item 1 resistencia de  1k$\Omega$
	\item 2 resistencia de 470$\Omega$
	\item 2 resistencia de 330$\Omega$
	\item 1 condensador de 0.1$\mu$F
	\item 1 pecera o caja de acrílico transparente o de cartulina blanca
	\item 1 protoboard o circuito impreso equivalente
	\item 1 fuente de alimentación regulada a 5V y al menos 2 amperios de salida
	\item Cables y conectores varios
\end{itemize}

\footnotetext{La resistencia de 12k$\Omega$ puede reemplazarse con resistencias de 13k$\Omega$ a 20k$\Omega$ dependiendo del voltaje de los diodos.} %chktex 42
