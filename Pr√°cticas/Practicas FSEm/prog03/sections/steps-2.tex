% %% %%%%%%%%%%%%%%%%%%%%%%%%%%%%%%%%%%%%%%%%%%%%%%%%%%%%%%%%%%
% step-2.tex
%
% Author:  Mauricio Matamoros
% License: MIT
%
% %% %%%%%%%%%%%%%%%%%%%%%%%%%%%%%%%%%%%%%%%%%%%%%%%%%%%%%%%%%%

% CHKTEX-FILE 1
% CHKTEX-FILE 46
%!TEX ROOT=../main.tex
%!TEX ROOT=../references.bib

\subsection{Paso 2: Cálculo del divisor de voltaje}%
\label{sec:step2}

El simulador configurado en la \Cref{sec:step1} implementa la simulación de un sensor de temperatura LM35 en configuración básica acoplado a un circuito ADC con un divisor de voltaje en $V_{Ref+}$ y $V_{Ref-}$ a tierra, mismo que se configura con la línea:

\lstinputlisting[%
	firstline=58,
	lastline=58,
	label={lst:adc-board-config},
	caption=Configuración del ADC del sensor LM35 virtual \IIC{} (\texttt{temperature.py:58})
]{src/temperature.py}

La función \texttt{run\_temperature\_board} recibe tres parámetros, los valores de las resistencias $R_1$ y $R_2$ para la alimentación del $V_{Ref+}$ y un boleano que configura el módulo ADC para operar con una presición de 8 bits (\texttt{p8bits=True}) o de 10 bits  (\texttt{p8bits=False}).

Sin embargo, los valores por defecto de las resistencias usadas no son óptimos, ya que hacen que el módulo ADC opere en rango completo de $0-5V$ mientras que el LM35 en configuración básica opera en el rango de $0.02-1.50V$, motivo por el cual deben proporcionarse otros valores para las resistencias.

Para calcular las resistencias del divisor de voltaje, lea cuidadosamente la \Cref{sec:intro-adc} y utilice la \Cref{eqn:vdiv} tomando un valor fijo para $R_1$ (ej.~$10k\Omega$) y despejando $R_2$.

\begin{importantbox}{\bfseries IMPORTANTE}
Verifique que al calcular los valores de $R_1$ y $R_2$ aproxima sus resultados a los valores de resistencias comerciales, por ejemplo cambiando una resitencia de $1273.5\Omega$ por una resistencia comercial de $1k2$.
\end{importantbox}